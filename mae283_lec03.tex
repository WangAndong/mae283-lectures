\documentclass[lecture,12pt,]{pcms-l}
\input preamble.tex

% For faster processing, load Matlab syntax for listings
\definecolor{MyDarkGreen}{rgb}{0.0,0.4,0.0}
\lstloadlanguages{Matlab}%
\lstset{language=Matlab,
        frame=single,
        basicstyle=\small\ttfamily,
        keywordstyle=[1]\color{Blue}\bf,
        keywordstyle=[2]\color{Purple},
        keywordstyle=[3]\color{Blue}\underbar,
        identifierstyle=,
        commentstyle=\usefont{T1}{pcr}{m}{sl}\color{MyDarkGreen}\small,
        stringstyle=\color{Purple},
        showstringspaces=false,
        tabsize=5,
        % Put standard MATLAB functions not included in the default
        % language here
        morekeywords={xlim,ylim,var,alpha,factorial,poissrnd,normpdf,normcdf},
        % Put MATLAB function parameters here
        morekeywords=[2]{on, off, interp},
        % Put user defined functions here
        morekeywords=[3]{FindESS},
        morecomment=[l][\color{Blue}]{...},
        numbers=left,
        firstnumber=1,
        numberstyle=\tiny\color{Blue},
        stepnumber=0
        }

% Only the next five fields need to be edited.
\newcommand{\lecAuth}{R.A. de Callafon}
\newcommand{\scribe}{Thomas Denewiler}
\newcommand{\authEmail}{callafon@ucsd.edu}
\newcommand{\scribeEmail}{tdenewiler@gmail.com}
\newcommand{\course}{MAE 283: Parameter Estimation}
\newcommand{\lectureNum}{3}

\address{Department of Mechanical and Aerospace Engineering, University of California, San Diego}

% Adds a hyperlink to an email address.
\newcommand{\mailto}[2]{\href{mailto:#1}{#2}}

% These commands set the document properties for the PDF output. Needs the hyperref package.
\hypersetup
{
    colorlinks,
    linkcolor={black},
    citecolor={black},
    filecolor={black},
    urlcolor={black},
    pdfauthor={\scribe <\mailto{\scribeEmail}{\scribeEmail}>},
    pdfsubject={\course},
    pdftitle={Lecture \lectureNum},
    pdfkeywords={UC San Diego, Parameter Estimation, System Identification},
    pdfstartpage={1},
}

% Includes a figure
% The first parameter is the label, which is also the name of the figure
%   with or without the extension (e.g., .eps, .fig, .png, .gif, etc.)
%   IF NO EXTENSION IS GIVEN, LaTeX will look for the most appropriate one.
%   This means that if a DVI (or PS) is being produced, it will look for
%   an eps. If a PDF is being produced, it will look for nearly anything
%   else (gif, jpg, png, et cetera). Because of this, when I generate figures
%   I typically generate an eps and a png to allow me the most flexibility
%   when rendering my document.
% The second parameter is the width of the figure normalized to column width
%   (e.g. 0.5 for half a column, 0.75 for 75% of the column)
% The third parameter is the caption.
\newcommand{\scalefig}[3]{
  \begin{figure}[ht!]
    % Requires \usepackage{graphicx}
    \centering
	\fbox{
	    \includegraphics[width=#2\columnwidth]{#1}
	}
    %%% I think \captionwidth (see above) can go away as long as
    %%% \centering is above
    %\captionwidth{#2\columnwidth}%
    \caption{#3}
    \label{#1}
  \end{figure}}

% Includes a MATLAB script.
% The first parameter is the label, which also is the name of the script
%   without the .m.
% The second parameter is the optional caption.
\newcommand{\matlabscript}[2]
  {\begin{itemize}\item[]\lstinputlisting[caption=#2,label=#1]{#1.m}\end{itemize}}

% A command to show a vector norm that will have the pipe signs scale with the contents.
\newcommand{\vectornorm}[1]{\left|\left|#1\right|\right|}

% Commands for time and frequency integrals over infinty, cos and sin.
\newcommand{\tint}{\int_{t=-\infty}^\infty}
\newcommand{\fint}{\int_{\omega=-\infty}^\infty}
\newcommand{\tauint}{\int_{\tau=0}^\infty}
\newcommand{\w}{\omega}
\newcommand{\wo}{\omega_0}
\newcommand{\ejwt}{e^{j\omega t}}
\newcommand{\emjwt}{e^{-j\omega t}}
\newcommand{\dt}{\Delta T}


%%%%%%%%%%%%%%%%%%%%%%%%%%%%%%%%%%%%%%%%%%%%%%%%%%%%%%%%%%%%%


\begin{document}
\mainmatter
\setcounter{page}{1}

\lectureseries[\course]{\course}

\auth[R.A. de Callafon]{Lecturer: \lecAuth\\ Scribe: \scribe}
\date{October 1, 2009}

\setaddress

% the following hack starts the lecture numbering at 3
\setcounter{lecture}{2}
\setcounter{chapter}{2}

\lecture{Discrete Time Systems}

\section{System Descriptions}
Most of the work done for this course will be in discrete time as a consequence of dealing with sampled signals, but it is relatively easy to convert back to continuous time with some minor assumptions.

\subsection{Continuous Time}
Continuous time systems are described by
$$y(t) = \tauint g(\tau)u(t-\tau)d\tau$$
where $g(\tau)=\mathcal{L}^{-1}\lbrace G(s)\rbrace$ and $y(s)=G(s)u(s)$. Note that multiplication in the Laplacian domain is equivalent to convolution in the time domain.

\subsection{Discrete Time}
For discrete time it is useful to model the input signal, $u(k\Delta T)$, as a ``Zero-order hold'' (ZOH) signal. *** Refer to a figure here. *** Other methods of modeling the input signal include linear fit where the discrete points are connected by lines and polynomial fitting, but those methods don't help nearly as much with the math we will see later in this lecture. What happens to the input when $u(t)$ is ZOH?
\begin{align*}
y(t) &= \tauint g(\tau)u(k\dt -\tau)d\tau \\
&= \sum_{l=0}^\infty \int_{\tau=(l-1)\dt}^{l\dt}g(\tau)u(k\dt -\tau)d\tau \\
&= \sum_{l=0}^\infty g_d(l)u_d(k-l) \\
&= \sum_{l=0}^\infty q^{-1}u(k)
\end{align*}
where $g_d(l)=\int_{\tau=(l-1)\dt}^{l\dt}g(\tau)u(k\dt -\tau)d\tau$. The second equality is due to our ability to consider $u(k\dt -\tau)$ to be constant. Note that the expressions $qu(k) = u(k+1)$ and $q^{-1}u(k)=u(k-1)$ are not multiplication operations but instead are time-shift operators. All of this results in
$$y(k) = \sum_{l-0}^\infty g_k(l)q^{-l}u(k) = G(q)u(k)$$
which shows that the output is a linear combination of \textit{all} past inputs.


\end{document}

%%%%%%%%%%%%%%%%%%%%%%%%%%%%%%%%%%%%%%%%%%%%%%%%%%%%%%%%%%%%%