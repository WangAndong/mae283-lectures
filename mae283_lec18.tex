%%%%%%%%%%%%%%%%%%%%%%%%%%%%%%%%%%%%%%%%%%%%%%%%%%%%%%%%%%%%%

\mainmatter
\setcounter{page}{1}

\lectureseries[\course]{\course}

\auth[\lecAuth]{Lecturer: \lecAuth\\ Scribe: \scribe}
\date{December 1, 2009}

\setaddress

% the following hack starts the lecture numbering at 18
\setcounter{lecture}{17}
\setcounter{chapter}{17}

\lecture{Numerical Methods}

\section{Recursive Estimation}
This lecture roughly corresponds to Chapters 10.1 and 10.2 of Ljung. These are iterative solutions for non-linear optimization. They also lead to iterative/recursive algorithms that can be used to estimate the parameters of a system. The algorithms are mostly used in the areas of adaptive control and fault detection. They are also known as online parameter estimation techniques.

\section{Newton-Raphson}
This is a quick derivation of the Newton-Raphson method where we are looking for a solution to the equation $f(\theta)=0$ which is given by
$$\hat{\theta} = \argsol_\theta f(\theta) = 0$$
Figure ?? shows an example function where we would be looking for the root of the function. This is similar to the Instrumental Variable (IV) method we saw in Lecture \ref{lec:IV} where the solution we obtained was given by
$$\hat{\theta}_{IV} = \argsol_\theta f(\theta,Z^N) = 0$$
In the IV method $f$ is linearly parameterized in $\theta$ so it is easy to compute a solution such that
$$\hat{\theta}_{IV} = [R(N)]^{-1}[f(N)]$$
The Newton-Raphson algorithm is developed by looking at
$$f(\theta_i), \qquad \left.f^\prime(\theta)\right|_{\theta=\theta_i}$$
This gives an update formula as
\begin{align*}
\theta_{i+1}: f(\theta_i) + (\theta_{i+1}-\theta_i)\left.f^\prime(\theta)\right|_{\theta=\theta_i} = 0
\end{align*}
\begin{align*}
\boxed{\theta_{i+1} = \theta_i - \left[\left.f^\prime(\theta)\right|_{\theta=\theta_i}\right]^{-1}\cdot f(\theta_i)}
\end{align*}
We can make this slightly more general by using
\begin{align*}
\boxed{\theta_{i+1} = \theta_i - \alpha(\theta_i)\left[\left.f^\prime(\theta)\right|_{\theta=\theta_i}\right]^{-1}\cdot f(\theta_i)}
\end{align*}
where $\alpha(\theta_i)$ is such that $|f(\theta_{i+1})|<|f(\theta_i)|$ and $\alpha(\theta_i)$ is known as the bending parameter or step-size function.

The Newton-Raphson method can also be used to find the minimum of a function as in Figure ??. Then the algorithm would be
\begin{align*}
\boxed{\theta_{i+1} = \theta_i - \alpha(\theta_i)\left[\left.f^{\prime\prime}(\theta)\right|_{\theta=\theta_i}\right]^{-1}\cdot\left[\left.f^\prime(\theta)\right|_{\theta=\theta_i}\right]}
\end{align*}
Note that we are looking for a minimum so we want $f^{\prime\prime}(\theta)>0$ and $f^{\prime\prime}(\theta)$ is known as the Hessian.

\subsection{Gradient Method}
What if $f^{\prime\prime}(\theta)$ is unknown? Then we can use the gradient method where the algorithm is
\begin{align*}
\boxed{\theta_{i+1} = \theta_i - \alpha(\theta_i)\cdot \left.f^\prime(\theta)\right|_{\theta=\theta_i}}
\end{align*}
where $\alpha(\theta_i)$ is such that $f(\theta_{i+1})<f(\theta_i)$.

\subsection{Convergence}
Suppose $f(\theta)$ is quadratic in $\theta$ where $f(\theta)=a+b\theta+c\theta^2$. It only takes one step for the Newton-Raphson method to converge to the minimum because it has quadratic convergence properties. Note that the algorithm converges to the least squares solution.

\section{More General Algorithm}
This is for cases where a least squares solution is not the best approach. Given
\begin{align*}
\hat{\theta}_N &= \argmin_\theta V_N(\theta) \\
V_N(\theta) &= \fN\sumt\est \\
\ett &= H_\theta^{-1}(Y(t)-G_\theta u(t))
\end{align*}
and using an ARX model
$$\ett = A_\theta y(t)-B_\theta u(t)$$
a more general update equation is given by
\begin{align*}
\boxed{\hat{\theta}_N^{i+1} = \hat{\theta}_N^i + \alpha(\hat{\theta}_N^i)\cdot F(\hat{\theta}_N^i)}
\end{align*}
where $\alpha(\cdot)$ is a step-size function and $F(\cdot)$ is the search direction. Obvious choices include
\begin{align*}
F(\hat{\theta}_N^i) &= -\frac{\partial}{\partial\theta}V_N(\theta) \\
F(\hat{\theta}_N^i) &= -\left[\frac{\partial^2}{\partial\theta^2}V_N(\theta)\right]^{-1}\cdot\left[\frac{\partial}{\partial\theta}V_N(\theta)\right]
\end{align*}
where the first equation is the gradient method and the second equation is the Newton-Raphson method. Note that the Newton-Raphson converges quickly but requires knowledge of the second derivative while the gradient method is easier to compute but less efficient.

\section{General Gradient Method}
Define
$$\psi(t,\theta) = \frac{\partial}{\partial\theta}\ett = -\frac{\partial}{\partial\theta}\hat{y}(t~|~t-1,\theta)$$
where $\hat{y}$ is the one-step ahead predictor and $\theta,\phi\in\mathbb{R}^{p\times1}$. This leads to
$$G_N(\theta) = \frac{\partial}{\partial\theta}V_N(\theta) = \fN\sumt\psi(t,\theta)\ett$$
Now we can see that the gradient is the auto-correlation of $\psi$ and $\eps$, $R_{\psi\eps}(0)$, and at the local minimum we have $R_{\psi\eps}(0)=0$. Next we set up matrices with these values such that
\begin{align*}
\Psi_\theta &= \frac{1}{\sqrt{N}} [\psi(1,\theta) ~ \psi(2,\theta) ~ \cdots ~ \psi(N,\theta)]^T\in\mathbb{R}^{N\times P} \\
\mathcal{E}_\theta &= \frac{1}{\sqrt{N}} [\eps(1,\theta) ~ \eps(2,\theta) ~ \cdots ~ \eps(N,\theta)]^T\in\mathbb{R}^{N\times1}
\end{align*}
\begin{align*}
\boxed{\theta_{i+1} = \theta_i + \alpha(\theta_i)\Psi_\theta^T\mathcal{E}_\theta}
\end{align*}
Here $\Psi_\theta$ is the regressor and we stop iterating when $\Psi_\theta^T\mathcal{E}_\theta=0$, which is called a stationary point.

\section{Gauss-Newton}
For the general case we have
\begin{align*}
\frac{\partial^2}{\partial\theta^2}V_N(\theta) &= \fN\sumt\psi(t,\theta)\psi^T(t,\theta) - \fN\sumt\left[\frac{\partial}{\partial\theta}\psi(t,\theta)\right]\ett
\end{align*}
which results in a $d\times d$ matrix if $\psi(t,\theta)$ is a $d\times1$ vector. We can approximate this equation by setting the last term to zero to get
\begin{align}
\label{eq:18hn}
\frac{\partial^2}{\partial\theta^2}V_N(\theta) &\approx H_N(\theta) = \fN\sumt\psi(t,\theta)\psi^T(t,\theta) \\
\label{eq:18hnalt}
H_N(\theta) &= \left[\fN\sumt\psi(t,\theta)\psi^T(t,\theta) + \eta I_{d\times d}\right]
\end{align}
where the term $\eta I_{d\times d}$ is known as a regularization.

The motivation for ignoring the last term but adding a constant is
\begin{itemize}
\item Avoids the computation of
$$\frac{\partial}{\partial\theta}\psi(t,\theta) = \frac{\partial^2}{\partial\theta^2}\ett$$
\item It is close to the minimum
$$\fN\sumt\frac{\partial}{\partial\theta}\psi(t,\theta)\ett \approx 0$$
\item $H_N(\theta)$ in (\ref{eq:18hn}) might not be invertible but $H_N(\theta)$ in (\ref{eq:18hnalt}) will always be positive definite and the inverse will exist.
\end{itemize}
This form of the udpate equation is known as the damped Gauss-Newton method and the algorithm is given by
\begin{align*}
\boxed{\theta_{i+1} = \theta_i + \alpha(\theta_i)\left[\Psi_\theta^T\Psi_\theta\right]^{-1} \left[\Psi_\theta\mathcal{E}_\theta\right]}
\end{align*}

\subsection{Comparison with Least Squares}
Recall that in least squares we had
\begin{align*}
\mathcal{M}: \ett &= y(t) - \vp^T(t)\theta \\
\frac{\partial}{\partial\theta}\ett &= \psi(t,\theta) = \vp(t)
\end{align*}
The estimate does not depend on $\theta$! The iteration is
$$\theta_{i+1} = \theta_i + \alpha(\theta_i)\left[\Phi^T\Phi\right]^{-1}\left[\Phi^T\mathcal{E}\right]$$
We can set $\theta_i=0$ and $\alpha(\theta_i)=1$ because we are free to choose the starting point and step-size function. This gives
$$\theta_{i+1} = \left[\Phi^T\Phi\right]^{-1}\left[\Phi^T\mathcal{E}\right]$$

\subsection{Effect of Regularization Parameter}
Recall that $\eta$ is the regularization parameter in the general Gauss-Newton method and that
$$\theta_{i+1} = \theta_i+\alpha(\theta_i)\left[\Phi_\theta^T\Phi_\theta + \eta I\right]^{-1}\left[\Phi_\theta\mathcal{E}_\theta\right]$$
We can see that there are two extremes such that
\begin{itemize}
\item $\eta=0$ results in pure Gauss-Newton method.
\item $\eta=1$ results in pure gradient method.
\end{itemize}
To get a simpler and more efficient result we can use the matrix inversion lemma to get
$$\left[\Lambda+\Psi^TW\Psi\right]^{-1} = \Lambda^{-1} - \Lambda^{-1}\Psi^T\left[\Psi\Lambda^{-1}\Psi^T + W^{-1}\right]^{-1}\Psi\Lambda^{-1}$$
Using $\Lambda=\eta I$ and $W=I$ gives
\begin{align*}
H_N^{-1} &= \left[\eta I + \Psi^T\Psi\right]^{-1} \\
&= \eta^{-1}-\eta^{-1}\Psi^T\left[\Psi\eta^{-1}\Psi^T+I\right]^{-1}\Psi\eta^{-1} \\
&= \eta^{-1}\left[I-\Psi^T\left[\Psi\Psi^T+\eta I\right]\Psi\right]
\end{align*}
The important bit is that the term $\Psi\Psi^T+\eta I$ is a scalar! We don't need the term $\alpha(\theta_i)$ because $\eta$ plays the same role.

Now, if we define $\gamma=\eta^{-1}$ and $\Gamma=\gamma\cdot I$ we get
\begin{align*}
\boxed{\theta_{i+1} = \theta_i - \Gamma\left[I+\Psi^T\Gamma\Psi\right]^{-1}\Psi^T\mathcal{E}}
\end{align*}
which is known as Gauss-Newton plus regularization.

We could use the matrix inversion lemma to get a scalar for the inverse again, which would give the result
\begin{align*}
\boxed{\theta_{i+1} = \theta_i - \underbrace{\frac{\gamma}{1+\Psi^T\gamma\Psi}}_{\text{adaptation gain}}\underbrace{\Psi^T\mathcal{E}}_{\text{gradient}}}
\end{align*}
This is a very easy value to compute numerically. The adaptation gain is also called the regularized adaptation gain. It can also be written as
$$\frac{\gamma}{1+\Psi^T\gamma\Psi} = \frac{\gamma}{1+\gamma\Psi^T\Psi}$$
This gain automatically adjusts for poor or good data.

\section{Conclusions}
In iterative parameter search using the regularized (damped) Gauss-Newton algorithm we have
$$\theta_{i+1} = \theta_i + \frac{\gamma(\theta_i)}{1+\Psi_{\theta_i}\gamma(\theta_i)\Psi_{\theta_i}^T}\cdot \Psi_{\theta_i}^T\mathcal{E}_{\theta_i}$$
which is a simple gain adjusted gradient method. We only need to compute
$$\Psi(t) = \frac{\partial}{\partial\theta}\ett$$

$Psi(t)$ is an $\mathbb{R}^{d\times1}$ signal and depends on the chosen model structure.

A stationary point is reached when $\Psi_\theta^T\mathcal{E}_\theta=0=$ the gradient.

We can control which numerical method is used with the regularization parameter where
\begin{itemize}
\item $\eta = \gamma^{-1} \gg 1$ or $\gamma\approx0$ is pure gradient method.
\item $\eta = \gamma^{-1} \ll 1$ or $\gamma\gg1$ is pure Gauss-Newton method.
\end{itemize}
%%%%%%%%%%%%%%%%%%%%%%%%%%%%%%%%%%%%%%%%%%%%%%%%%%%%%%%%%%%%%