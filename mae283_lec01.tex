\documentclass[lecture,12pt,]{pcms-l}
\input preamble.tex

% For faster processing, load Matlab syntax for listings
\definecolor{MyDarkGreen}{rgb}{0.0,0.4,0.0}
\lstloadlanguages{Matlab}%
\lstset{language=Matlab,
        frame=single,
        basicstyle=\small\ttfamily,
        keywordstyle=[1]\color{Blue}\bf,
        keywordstyle=[2]\color{Purple},
        keywordstyle=[3]\color{Blue}\underbar,
        identifierstyle=,
        commentstyle=\usefont{T1}{pcr}{m}{sl}\color{MyDarkGreen}\small,
        stringstyle=\color{Purple},
        showstringspaces=false,
        tabsize=5,
        % Put standard MATLAB functions not included in the default
        % language here
        morekeywords={xlim,ylim,var,alpha,factorial,poissrnd,normpdf,normcdf},
        % Put MATLAB function parameters here
        morekeywords=[2]{on, off, interp},
        % Put user defined functions here
        morekeywords=[3]{FindESS},
        morecomment=[l][\color{Blue}]{...},
        numbers=left,
        firstnumber=1,
        numberstyle=\tiny\color{Blue},
        stepnumber=0
        }

% Only the next five fields need to be edited.
\newcommand{\lecAuth}{R.A. de Callafon}
\newcommand{\scribe}{Thomas Denewiler}
\newcommand{\authEmail}{callafon@ucsd.edu}
\newcommand{\scribeEmail}{tdenewiler@gmail.com}
\newcommand{\course}{MAE 283: Parameter Estimation}
\newcommand{\lectureNum}{1}

\address{Department of Mechanical and Aerospace Engineering, University of California, San Diego}

% Adds a hyperlink to an email address.
\newcommand{\mailto}[2]{\href{mailto:#1}{#2}}

% These commands set the document properties for the PDF output. Needs the hyperref package.
\hypersetup
{
    colorlinks,
    linkcolor={black},
    citecolor={black},
    filecolor={black},
    urlcolor={black},
    pdfauthor={\scribe <\mailto{\scribeEmail}{\scribeEmail}>},
    pdfsubject={\course},
    pdftitle={Lecture \lectureNum},
    pdfkeywords={UC San Diego, Parameter Estimation, System Identification},
    pdfstartpage={1},
}

% Includes a figure
% The first parameter is the label, which is also the name of the figure
%   with or without the extension (e.g., .eps, .fig, .png, .gif, etc.)
%   IF NO EXTENSION IS GIVEN, LaTeX will look for the most appropriate one.
%   This means that if a DVI (or PS) is being produced, it will look for
%   an eps. If a PDF is being produced, it will look for nearly anything
%   else (gif, jpg, png, et cetera). Because of this, when I generate figures
%   I typically generate an eps and a png to allow me the most flexibility
%   when rendering my document.
% The second parameter is the width of the figure normalized to column width
%   (e.g. 0.5 for half a column, 0.75 for 75% of the column)
% The third parameter is the caption.
\newcommand{\scalefig}[3]{
  \begin{figure}[ht!]
    % Requires \usepackage{graphicx}
    \centering
	\fbox{
	    \includegraphics[width=#2\columnwidth]{#1}
	}
    %%% I think \captionwidth (see above) can go away as long as
    %%% \centering is above
    %\captionwidth{#2\columnwidth}%
    \caption{#3}
    \label{#1}
  \end{figure}}

% Includes a MATLAB script.
% The first parameter is the label, which also is the name of the script
%   without the .m.
% The second parameter is the optional caption.
\newcommand{\matlabscript}[2]
  {\begin{itemize}\item[]\lstinputlisting[caption=#2,label=#1]{#1.m}\end{itemize}}

% A command to show a vector norm that will have the pipe signs scale with the contents.
\newcommand{\vectornorm}[1]{\left|\left|#1\right|\right|}


%%%%%%%%%%%%%%%%%%%%%%%%%%%%%%%%%%%%%%%%%%%%%%%%%%%%%%%%%%%%%


\begin{document}
\mainmatter
\setcounter{page}{1}

\lectureseries[\course]{\course}

\auth[\lecAuth]{Lecturer: \lecAuth\\ Scribe: \scribe}
\date{September 24, 2009}

\setaddress

% the following hack starts the lecture numbering at 1
\setcounter{lecture}{0}
\setcounter{chapter}{0}

\lecture{Course Introduction}

\section{Introduction}
This lecture is an introduction to the course. Typical lectures will be done at the blackboard but this one is done using slides that can be found at the main course website -- \href{http://mechatronics.ucsd.edu/mae283a/}{http://mechatronics.ucsd.edu/mae283a/}. The main contact information for Prof. de Callafon is \mailto{\authEmail}{\authEmail}. The grading will consist of the average of three homework assignments (20\%), one in-class midterm (30\%) and a final (50\%), where the final is split between a take-home project and an in-class test. The main text is \href{http://www.amazon.com/s/ref=nb\_ss?url=search-alias\%3Daps\&field-keywords=0-13-656695-2&x=0\&y=0}{``System Identification, Theory for the User''} by Lennart Ljung, Second Edition (ISBN: 0-13-656695-2) and the recommended text is \href{http://www.amazon.com/s/ref=nb\_ss?url=search-alias\%3Daps\&field-keywords=0-13-881236-5&x=0\&y=0}{``System Identification''} by Torsten Soderstrom and Petre Stoica (ISBN: 0-13-881236-5). The slides, pictures and videos for this lecture can be found at the course website.

\section{What Is System Identification?}
System identification is the modeling of a dynamic system on the basis of experimental data. It can be thought of as the formulation and estimation of systematic relationships between measured discrete-time input/output data. Formulation is the parameterization of a dynamic model, or determining the salient parameters necessary to model a system. Estimation is the realization and/or optimization of those parameters. Discrete time concerns the sampling (multivariable) of continuous time systems at consistent or known intervals. Input/output gives the causal relationship between measurable actions and the reaction of the dynamic system.

System identification can also be seen as a systematic method of data extraction techniques to reduce a large number $N$ of noisy I/O observations to a small number $p \ll N$ of parameters of a deterministic or stochastic dynamic model. The model that is produced is referred to as the ``empirical'' or ``data-based'' model. It is an alternative to ``physics'' or ``first-principle'' methods of generating models. Models based on the laws of physics tend to be much more complex due to the greater number of parameters and are thus more likely to have greater uncertainties. The models generated using system identification sacrifice capturing higher-order dynamics compared to first-principle-based models but are generally much simpler and can have performance criteria as constraints so that enough of the dynamics can be modeled to make the model useful.

\end{document}

%%%%%%%%%%%%%%%%%%%%%%%%%%%%%%%%%%%%%%%%%%%%%%%%%%%%%%%%%%%%%